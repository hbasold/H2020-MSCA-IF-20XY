% !TEX root = ./IF-2018-Part_B.tex

\newpage
\addcontentsline{toc}{section}{\hspace{-0.5cm}Document 2}
\section{CV of the Experienced Researcher}
\label{sec:cv}

The CV is intrinsic to the evaluation of the whole proposal and is assessed throughout the 3 evaluation criteria by the expert evaluators.
Please make sure that the information between part A and B is fully consistent.

\medskip\noindent
Applicants without a doctorate should clearly justify any period of Full-Time
Equivalent Research Experience in the CV part B (section~\ref{sec:cv}). It is essential that the
CV clearly explains how the Research Experience is calculated, following this
template.

\medskip\noindent
This section should be limited to maximum 5 pages and should include \textbf{the standard academic and research record}. 
Any research career gaps and/or unconventional paths should be clearly explained so that this can be fairly assessed by the independent evaluators.

\medskip\noindent
The {\em experienced researchers} must provide a list of achievements reflecting their track record, if applicable:

\begin{enumerate}
\item \textbf{Publications} in peer-reviewed scientific journals, peer-reviewed conference proceedings and/or monographs of their respective research fields, indicating also the number of citations (excluding self-citations) they have attracted.
\item Granted \textbf{patent(s)}.
\item \textbf{Research monographs, chapters} in collective volumes and any translations thereof.
\item \textbf{Invited presentations} to peer-reviewed, internationally established conferences and/or international advanced schools.
\item \textbf{Research expeditions} led by that the {\em experienced researcher}. 
\item \textbf{Organisation of International conferences} in the field of the researcher (membership in the steering and/or programme committee).
\item Examples of \textbf{participation in industrial innovation}.
\item \textbf{Prizes and Awards}.
\item \textbf{Funding} received so far.
\item \textbf{Supervising} and \textbf{mentoring} activities.
\end{enumerate}





\newpage
\section{Capacity of the Participating Organisations}
\label{sec:capacities}

Beneficiaries and partner organisations must complete the appropriate table below.

\medskip\noindent
Complete one table (min font size: 9) of maximum \ul{one page per beneficiary and one page per partner organisation}. 
The expert evaluators will be instructed to disregard content above this limit.
\vspace{\baselineskip}


\begin{table}[h!]
{\fontsize{9bp}{1em}\selectfont % should be 9pt
\noindent\begin{tabular}{|>{\raggedright}p{.25\textwidth}|p{.7\textwidth}|}\hline
  \multicolumn{2}{|l|}{\cellcolor{gray!50}\textbf{Beneficiary X}} \\\hline
\textbf{General Description} &

\\\hline
\textbf{Role and Profile of key persons (supervisor)} &
{\em (names, title, qualifications of the main supervisor)}
{\em }
\\\hline
\textbf{Key Research Facilities, Infrastructure and Equipment} &
{\em Demonstrate that the beneficiary has sufficient facilities and infrastructure to host and/or offer a suitable environment for training and transfer of knowledge to the recruited experienced researcher.
\newline
If applicable, indicate the name of the entity with a capital or
legal link to the beneficiary and its role in the action.
}
\\\hline
\textbf{Independent research premises?} &
{\em Please explain the status of the beneficiary's research facilities\----i.e. are they owned by the beneficiary or rented by it? Are its research premises wholly independent from other entities?
\newline
If applicable, indicate the name of the entity with a capital or
legal link to the beneficiary and describe the nature of the
link.
}
\\\hline
\textbf{Previous Involvement in Research and Training Programmes} &
{\em Detail any (maximum 5) relevant EU, national or international research and training actions/projects in which the beneficiary has previously participated.}
\\\hline
\textbf{Current involvement in Research and Training Programmes} &
{\em Detail the EU and/or national research and training actions in which the beneficiary is currently participating.}
\\\hline
\textbf{Relevant Publications and/or research/innovation products} &
{\em (Max 5) Only list items (co-)produced by the supervisor}
\\\hline
  \multicolumn{2}{|l|}{\cellcolor{gray!50}\textbf{Partner Organisation Y}} \\\hline
\textbf{General description} &
\\\hline
\textbf{Key Persons and Expertise (supervisor)} &
\\\hline
\textbf{Key Research facilities, infrastructure and equipment} &
\\\hline
\textbf{Previous and Current Involvement in Research and Training Programmes} &
\\\hline
\textbf{Relevant Publications and/or research/innovation product} &
{\em (Max 3)}
\\\hline
\end{tabular}}
\end{table}




\newpage
\section{Ethical Issues}
\label{sec:ethics}

Compliance with the relevant ethics provisions is essential from the beginning to the end of the action and is an integral part of research funded by the European Union within Horizon 2020. 

\medskip\noindent
Applicants submitting research proposals for funding with Marie Sk\l{}odowska-Curie actions in Horizon 2020 should demonstrate proactively that they are aware of and will comply with European and national legislation and fundamental ethical principles, including those reflected in the \href{http://www.europarl.europa.eu/charter/pdf/text_en.pdf}{Charter of Fundamental Rights of the European Union} and the \href{http://www.echr.coe.int/Documents/Convention_ENG.pdf}{European Convention on Human Rights and its Supplementary Protocols}.

\medskip\noindent
Please be aware that it is the applicants' responsibility to identify any potential ethical issue, 
to handle the ethical aspects of the proposal and to detail how these aspects will be addressed.

\bigskip\noindent
{\large {\bf \ul{The Ethics Review Procedure in Horizon 2020}}}

\medskip\noindent
All proposals above threshold and considered for funding will undergo an Ethics Review carried out by independent ethics experts. 
When submitted a proposal to Horizon 2020, all applicants are required to complete an ``{\bf Ethics Issues Table (EIT)}'' in the Part A of the proposal. 
Applicants who flag ethical issues in the EIT have to also complete a more in-depth {\bf Ethics Self-Assessment in Part B.}

\medskip\noindent
The ethics self-assessment will become part of the Grant Agreement and may thus lead to binding obligations that may later on be checked by ethics checks, reviews and audits.

\medskip\noindent
For more details, please refer to the H2020 \href{http://ec.europa.eu/research/participants/data/ref/h2020/grants_manual/hi/ethics/h2020_hi_ethics-self-assess_en.pdf}{``How to complete your Ethics Self-Assessment''} guide.

\bigskip\noindent
{\large {\bf \ul{Ethics Self-Assessment (Part B)}}}

\medskip\noindent
The Ethics Self-Assessment must:

{\bf
\begin{enumerate}[leftmargin=*, label=\arabic*)]
  \item Describe how the proposal meets the EU and national legal and ethics requirements of the country/countries where the task raising ethical is to be carried out.
\end{enumerate}
}

\medskip\noindent
For more information on how to deal with Third Countries (in the context of ethics
appraisal, Third Country refers to non-EU country; Associated Countries are "ethics"
TC) please see Article 34 of the \href{http://ec.europa.eu/research/participants/data/ref/h2020/grants_manual/amga/h2020-amga_en.pdf}{Annotated Model Grant Agreement},
as well as the following \href{http://ec.europa.eu/justice/data-protection/international-transfers/adequacy/index_en.htm}{link}.
Please ensure and confirm that the research performed outside the EU
is compatible with the Union, National and International legislation and could have
been legally conducted in one of the EU Member States.

\medskip\noindent
Please list the documents provided with their expiry date.

\medskip\noindent
Ensure early compliance of the proposed research with EU and national legislation on ethics in research.
Should your proposal be selected for funding, 
you will be required to confirm that you have obtained the following documents (if applicable):

\begin{enumerate}[label=(\alph*)]
  \item any ethics committee opinion required under national law and
  \item any notification or authorisation for activities raising ethical issues required under national and/or European law
\end{enumerate}

\noindent
needed for implementing the action tasks in question.

\medskip\noindent
If you have not already applied for/received the ethics approval/required ethics documents when submitting the proposal, 
please indicate in this section the approximate date when you will obtain the missing approval/any other ethics documents. 
Please state explicitly that you will not proceed with any research with ethical implications before 
obtaining the necessary authorizations/opinions.

\medskip\noindent
\fcolorbox{black}{gray!50}{\parbox{\textwidth}{
{\em The documents must be kept on file and be submitted upon request by the beneficiary
to the Agency (see Article 52). If they are not in English, they must be submitted
together with an English summary, which shows that the action tasks in question are
covered and includes the conclusions of the committee or authority concerned (if
available).}

\medskip\noindent
{\em If you plan to request these ethics documents specifically for your} proposed action\em{, your request must contain an explicit reference to the} action\em{'s title.}
}}

\bigskip
{\bf 
\begin{enumerate}[leftmargin=*, label=\arabic*), start=2]
  \item Explain in detail how you intend to address the ethical issues flagged, in particular with regard to: 
  \begin{itemize}
    \item {\normalfont the research {\bf objectives} (e.g., study of vulnerable populations, cooperation with a Third Country, etc.);}
    \item {\normalfont the research {\bf methodology} (e.g., clinical trials, involvement of children and related information and consent/assent procedures, data protection and privacy issues related to data collected, etc.);}
    \item {\normalfont the potential {\bf impact} of the research (e.g. dual use issues, environmental damage, malevolent use, etc.);}
    \item {\normalfont appropriate health and safety procedures - conforming to relevant local/national guidelines/legislation - for the staff involved;}
    \item {\normalfont possible harm to the environment the research might cause, (as an example: environmental risks of nanomaterials), and measures that will be taken to mitigate the risks.}
  \end{itemize}
\end{enumerate}
}




\newpage
\section{Letter of Commitment (GF only)}
\label{sec:letters}

For the Global Fellowship proposals, a letter of Commitment \textbf{of the partner
organisations} (hosting the outgoing phase in a third country) must be included in part
B-2 to ensure their real and active participation. these should not be attached as a
separate PDF file or as an embedded file since this makes them invisible.

\medskip\noindent
GF Proposals which fail to include a letter of commitment of the partner organisation
will be declared \textbf{inadmissible}.

\medskip\noindent
Minimum requirements for the letter of commitment: 

\begin{itemize}
  \item \textbf{heading} or \textbf{stamp} from the institution; 
  \item up-to-date (may not be dated prior to the call publication); 
  \item the text must demonstrate the will to actively participate in the (identified) proposed action and the precise role;
\end{itemize}

\noindent
Please note that no template for these letters is provided, only general rules.





\newpage
\label{sec:endpage}
\vspace{15mm}
\begin{center}


        \Large{
      
     
        \textbf{ENDPAGE}
  
          \vspace{15mm}
          MARIE SK\L{}ODOWSKA-CURIE ACTIONS\\
          \vspace{1cm}
          
          \textbf{\acf{IF}}\\
          \textbf{Call: H2020-MSCA-IF-2018}
          \vspace{2cm}                   

          PART B
          \vspace{2.5cm}

          ``{\sc \ac{PropAcronym}\xspace}''
          \vspace{2cm}

          \textbf{This proposal is to be evaluated as:}
          \vspace{.5cm}

          \textbf{[EF-ST] [EF-CAR] [EF-RI] [EF-SE] [GF]}\\
        }
        \large{[Delete as appropriate]}

  \end{center}
\vspace{1cm}
