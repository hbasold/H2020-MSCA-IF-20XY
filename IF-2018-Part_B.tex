\documentclass[a4paper,11pt]{article}

\usepackage[T1]{fontenc}
\usepackage[utf8]{inputenc}
\usepackage{lmodern}
\usepackage{eurosym}
\usepackage{lastpage}
\usepackage{xspace}
\usepackage[margin=15mm,includehead,includefoot]{geometry}
\usepackage{fancyhdr}
\usepackage{booktabs}
\usepackage{graphicx}
\usepackage{multirow}
\usepackage{array}
\usepackage[usenames,dvipsnames,table]{xcolor}
\usepackage{csquotes}
\usepackage{pgfgantt}
\usepackage[nohyperlinks,nolist]{acronym}

\usepackage[compact]{titlesec}
% HotFix from http://tex.stackexchange.com/a/300259/84485
% Version1 of titlesec is not compatible with the latest texlive. 
% Either the titlesec package must be updated, or the following HotFix used:
\usepackage{etoolbox}
\makeatletter
\patchcmd{\ttlh@hang}{\parindent\z@}{\parindent\z@\leavevmode}{}{}
\patchcmd{\ttlh@hang}{\noindent}{}{}{}
\makeatother

\usepackage{enumitem}
\usepackage[style=verbose-ibid,backend=bibtex]{biblatex}
\usepackage{hyperref}
\usepackage{amsfonts,amsmath,amssymb}
\usepackage{soul} % for smarter (word-wrapping) underlining.
\setul{1pt}{.4pt} % 1pt below contents
\usepackage{lineno}
\usepackage{enumitem} % for broken enumerate in ethics section.

\usepackage{helvet}
\renewcommand{\familydefault}{\sfdefault}

\newcommand{\TODO}[1]{{\textcolor{red}{[\textbf{TODO:} #1]}}}
% \titlespacing\section{0pt}{6pt plus 4pt minus 2pt}{0pt plus 2pt minus 2pt}
% \titlespacing\subsection{0pt}{6pt plus 4pt minus 2pt}{0pt plus 2pt minus 2pt}
% \titlespacing\subsubsection{0pt}{6pt plus 4pt minus 2pt}{0pt plus 2pt minus 2pt}

\titleformat*{\section}{\large\bfseries}
\titleformat*{\subsection}{\normalsize\bfseries}
\titleformat*{\subsubsection}{\normalsize\bfseries}
\let\oldfootnotesize\footnotesize
\renewcommand{\footnotesize}{\fontsize{8bp}{1em}\selectfont}
\renewcommand{\cite}{\autocite} % citations in footnotes
\bibliography{bibliography}

\headheight=14pt

% Explicitly set footnote font size to match call (i.e., 8pt).
% Taken from http://tex.stackexchange.com/a/249422/84485
\makeatletter
\renewcommand\footnotesize{%
   \@setfontsize\footnotesize\@ixpt{8}%
   \abovedisplayskip 8\p@ \@plus2\p@ \@minus4\p@
   \abovedisplayshortskip \z@ \@plus\p@
   \belowdisplayshortskip 4\p@ \@plus2\p@ \@minus2\p@
   \def\@listi{\leftmargin\leftmargini
               \topsep 4\p@ \@plus2\p@ \@minus2\p@
               \parsep 2\p@ \@plus\p@ \@minus\p@
               \itemsep \parsep}%
   \belowdisplayskip \abovedisplayskip
}
\makeatother

\hypersetup{
    pdftitle={H2020-MSCA-IF-2018},    % title
    pdfauthor={},
    colorlinks=true,
    citecolor=black,
    linkcolor=black,
    urlcolor=blue
  }

\newcommand{\markStartPageLimit}{%
\begin{flushright}
\textcolor{red}{{\large {\bf START PAGE COUNT \--- MAX 10 PAGES~~~~~}}\\
\rule{\textwidth}{0.5mm}
}\vspace{-.5cm}
\end{flushright}
}
\newcommand{\markEndPageLimit}{%
\vfill
\begin{flushright}
\textcolor{red}{{\large {\bf STOP PAGE COUNT \--- MAX 10 PAGES~~~~~}}\\
\rule{\textwidth}{0.5mm}
}
\end{flushright}
}


% To correctly align fancy headers.
% Courtesy of: http://tex.stackexchange.com/a/88136/84485
\makeatletter
\newcommand{\resetHeadWidth}{\f@nch@setoffs}
\makeatother

\pagestyle{fancy}
\fancyhead{}
\fancyhead[C]{\color{gray}{\ac{PropAcronym}\xspace\xspace - EF-ST / EF-CAR / EF-RI / EF-SE / GF\\
(Delete as appropriate and include as header on each page)}}
\fancyfoot{}
\fancyfoot[C]{\color{gray}{Part B - Page \thepage~of \pageref*{sec:endpage}}}
\resetHeadWidth

\renewcommand{\headrulewidth}{0pt}


\renewcommand{\contentsname}{\centerline{TABLE OF CONTENTS}}
% The code below converts all section headings to upper case in the table of contents.
% Courtesy of: http://tex.stackexchange.com/a/156917/84485
\makeatletter
\let\oldcontentsline\contentsline
\def\contentsline#1#2{%
  \expandafter\ifx\csname l@#1\endcsname\l@section
    \expandafter\@firstoftwo
  \else
    \expandafter\@secondoftwo
  \fi
  {%
    \oldcontentsline{#1}{\MakeUppercase{#2}}%
  }{%
    \oldcontentsline{#1}{#2}%
  }%
}
\makeatother

%List of acronyms used in the proposal 
%read the acronym package manual for details: http://ctan.mackichan.com/macros/latex/contrib/acronym/acronym.pdf
%\acs{PropAcronym} gives acronym [in square brackets]
%\acl{PropAcronym} gives long form of acronym, probably not used {in second set of braces}
%\acf{PropAcronym} gives both
\begin{acronym}
\acro{IF}{Individual Fellowships}
\acro{PropAcronym}[PROPOSAL ACRONYM]{This is my proposal's acronym}
\end{acronym}

\begin{document}
%mark proposal acronym as used, such that the short-hand form is always used by default.
\acused{PropAcronym}
%\begin{linenumbers}

% !TEX root = ./IF-2018-Part_B.tex

\addcontentsline{toc}{section}{\hspace{-0.5cm}Document 1}
\addcontentsline{toc}{section}{Start Page}
\phantom{a}
\vspace{15mm}
\begin{center}


        \Large{
      
     
        \textbf{START PAGE}
  
          \vspace{15mm}
          MARIE SK\L{}ODOWSKA-CURIE ACTIONS\\
          \vspace{1cm}
          
          \textbf{\acf{IF}}\\
          \textbf{Call: H2020-MSCA-IF-2018}
          \vspace{2cm}                   

          PART B
          \vspace{2.5cm}

          ``{\sc \ac{PropAcronym}\xspace}''
          \vspace{2cm}

          \textbf{This proposal is to be evaluated as:}
          \vspace{.5cm}

          \textbf{[EF-ST] [EF-CAR] [EF-RI] [EF-SE] [GF]}\\
        }
        \large{[Delete as appropriate]}

  \end{center}
\vspace{1cm}

\newpage
\setcounter{tocdepth}{1}
\tableofcontents



\newpage
\markStartPageLimit
\section{Excellence}
\label{sec:excellence}
~\footnote{Literature should be listed in footnotes, font size 8 or 9.
All literature references will count towards the page limit.}

\subsection{Quality and credibility of the research/innovation action (level of novelty, appropriate consideration of inter/multidisciplinary and gender aspects)}
\label{sec:excellence_quality}

You should develop your proposal according to the following lines:
\begin{itemize}
  \item Introduction, state-of-the-art, specific objectives and overview of the action
  \item Research methodology and approach: highlight the type of research / innovation activities proposed
  \item Originality and innovative aspects of the research programme: explain the contribution that the action is expected to make to advancements within the action field. Describe any novel concepts, approaches or methods that will be implemented.
 \item The gender dimension in the research content (if relevant).

\setlength{\fboxsep}{1mm}
\fbox{\parbox{.98\linewidth}{
{\small In research activities where human beings are involved as subjects or end-users,
gender differences may exist. In these cases the gender dimension in the research
content has to be addressed as an integral part of the proposal to ensure the highest
level of scientific quality.}
}}

 \item The interdisciplinary aspects of the action (if relevant).
 \item Explain how the high-quality, novel research is the most likely to open up the best career possibilities for the {\em experienced researcher} and new collaboration opportunities for the host organisation(s). 
\end{itemize}




\subsection{Quality and appropriateness of the training and of the two way transfer of knowledge between the researcher and the host}
\label{sec:excellence_transfer}

Describe the training that will be offered.

\medskip\noindent
Outline how a two way transfer of knowledge will occur between the researcher and the host institution(s):
\begin{itemize}
\item Explain how the \emph{experienced researcher} will gain new knowledge during the fellowship at the hosting organisation(s).
\item Outline the previously acquired knowledge and skills that the researcher will transfer to the host organisation(s).
\end{itemize}

\noindent
For Global Fellowships explain how the newly acquired skills and knowledge in the Third Country will be transferred back to the host institution in Europe (the beneficiary) during the incoming phase.

\medskip\noindent
Typical \textbf{training activities} in Individual Fellowships may include:
\begin{itemize}
\item Primarily, \emph{training-through-research} by the means of an \ul{individual personalised
project}, under the guidance of the supervisor and other members of the research
staff of the host organisation(s)
\item Hands-on training activities for developing scientific skills (new techniques,
instruments, research integrity, 'big data'/'open science') and transferrable skills
(entrepreneurship, proposal preparation to request funding, patent applications,
management of IPR, project management, task coordination, supervising and
monitoring, take up and exploitation of research results)
\item Inter-sectoral or interdisciplinary transfer of knowledge (e.g. through secondments)
\item Taking part in the research and financial management of the action
\item Organisation of scientific/training/dissemination events
\item Communication, outreach activities and horizontal skills
\item Training dedicated to gender issues
\end{itemize}




\subsection{Quality of the supervision and of the integration in the team/institution}
\label{sec:excellence_supervision}

\begin{itemize}
  \item Qualifications and experience of the supervisor(s)

Provide information regarding the supervisor(s): 
the level of experience on the research topic proposed and their track record of work, 
including main international collaborations, 
as well as the level of experience in supervising/training especially at advanced level (PhD, postdoctoral) researchers.
Information provided should include participation in projects, publications, patents and any other relevant results.

  \item Hosting arrangements%
\footnote{The hosting arrangements refer to the integration of the researcher to his new environment in the premises of the host. 
It does not refer to the infrastructure of the host as described in the Quality and efficiency of the implementation criterion.}

The application must show that the experienced researcher will be well integrated within the team/institution in order that all parties gain the maximal knowledge and skills from the fellowship.
The nature and the quality of the research group/environment as a whole should be outlined, 
together with the measures taken to integrate the researcher in the different areas of expertise, disciplines, and international networking opportunities that the host could offer.

For GF both phases should be described - for the outgoing phase, specify the practical arrangements in place to host a researcher coming from another country, 
and for the incoming phase specify the measures planned for the successful (re)integration of the researcher.
\end{itemize}




\subsection{Capacity of the researcher to reach or re-enforce a position of professional maturity/independence}
\label{sec:excellence_maturity}

Applicants should \textbf{demonstrate} how their professional experience and the proposed
research will contribute to their development as independent/mature researchers, \ul{\textbf{during}}
the fellowship.

\medskip\noindent
Please keep in mind that the fellowships will be awarded to the most talented researchers
as shown by the proposed research and their track record (Curriculum Vitae, section 4),
in relation to their level of experience.

\medskip\noindent
A complete {\bf Career Development Plan should not be included in the proposal}, 
but it is part of implementing the action in line with the European Charter for Researchers. It
should aim at reaching a realistic and well-defined objective in terms of career
advancement (by attaining a leading independent position for example) or resuming a
research career after a break. The plan should be devised with the final outcome to
develop and significantly widen the competences of the experienced researcher,
particularly in terms of multi/interdisciplinary expertise, inter-sectoral experience and
transferable skills.




\newpage
\section{Impact}
\label{sec:impact}

\subsection{Enhancing the potential and future career prospects of the researcher}
\label{sec:impact_researcher}

Explain the expected impact of the planned research and training on the future career prospects of the experienced researcher \ul{\textbf{after}} the fellowship.

\medskip\noindent
Describe the added value of the fellowship on the future career opportunities of the
researcher.

\medskip\noindent
Which new competences and skills will be acquired? How should these make the
researcher more successful?




\subsection{Quality of the proposed measures to exploit and disseminate the action results}
\label{sec:impact_dissemination}

\setlength{\fboxsep}{1mm}
\fbox{\parbox{\textwidth}{
{\large {\bf Background -- Dissemination and exploitation of results}}

\medskip\noindent
Dissemination and Exploitation strategy is about the results of the action and it is targeted
at peers (scientific or the action's own community, industry and other commercial actors,
professional organisations, policymakers) and to the wider research and innovation
community - to achieve and expand the potential impact of the action. The proposal
should describe the foreseen dissemination and exploitation activities and their expected
impact.

\smallskip\noindent
All researchers should ensure, in compliance with their contractual arrangements, that the
results of their research are disseminated and exploited, e.g. communicated, transferred
into other research settings or, if appropriate, commercialised. Senior researchers, in
particular, are expected to take a lead in ensuring that research is fruitful and that results
are either exploited commercially or made accessible to the public (or both) whenever the
opportunity arises.

\smallskip\noindent
Please refer also to the \href{http://ec.europa.eu/research/participants/docs/h2020-funding-guide/grants/grant-management/dissemination-of-results_en.htm}{"Dissemination \& exploitation" section of the H2020 Online Manual}.
}}

\medskip\noindent
Describe how the new knowledge generated by the action will be disseminated and exploited, 
e.g. communicated, transferred into other research settings or, if appropriate, commercialised.
Describe, when relevant, how intellectual property rights will be dealt with.

\medskip\noindent
A concrete planning for section~\ref{sec:impact_dissemination} must be included in the Gantt Chart (see point~\ref{sec:implementation_work_plan}).




\subsection{Quality of the proposed measures to communicate the action activities to different target audiences}
\label{sec:impact_communication}

\setlength{\fboxsep}{1mm}
\fbox{\parbox{\textwidth}{
{\large {\bf Background -- Communication}}

\medskip\noindent
Communication of the action aims to demonstrate the ways in which the research,
training and mobility contribute to a European "Innovation Union" and account for public
spending. It should provide tangible proof that the funded action adds value by:

\begin{itemize}
\item showing how European and international collaboration has achieved more than would
have otherwise been possible, notably in achieving scientific excellence, contributing to
competitiveness and, where relevant, solving societal challenges;
\item showing how the outcomes are relevant to our everyday lives, by creating jobs, training
skilled researchers, introducing novel technologies, bringing ideas from research to
market or making our lives more comfortable in other ways;
\item promoting results, which may possibly influence policy-making, and ensure follow-up
by industry, civil society and by the scientific community.
\end{itemize}

In the MSCA, public engagement is an important part of communication. The primary
goal of public engagement activities is to create awareness among the general public of
the research work performed under these projects and its implications for citizens and
society. The type of outreach activities could range from press articles and participating
in European Researchers' Night events to presenting science, research and innovation
activities to students from primary and secondary schools or universities in order to
develop their interest in research careers.

\smallskip\noindent
Researchers should ensure that their research activities -- both the action and, when
available, its results -- are made known to society at large in such a way that they can be
understood by non-specialists, thereby improving the public's understanding of science.
Direct engagement with the public will help researchers to better understand public
interest in priorities for science and technology and also the public's concerns.

\bigskip\noindent
For more details, see the guide on \href{http://ec.europa.eu/research/participants/data/ref/h2020/other/gm/h2020-guide-comm_en.pdf}{Communicating EU research and innovation guidance
for project participants} as well as the \href{http://ec.europa.eu/research/participants/docs/h2020-funding-guide/grants/grant-management/communication_en.htm}{"communication" section of the H2020 Online
Manual}.
}}

\medskip\noindent
The frequency and nature of communication activities should be outlined in the proposal.
Concrete plans for the above must be included as a deliverable.

\medskip\noindent
A concrete planning for section~\ref{sec:impact_communication} must be included in the Gantt Chart (see point~\ref{sec:implementation_work_plan}).




\newpage
\section{Quality and Efficiency of the Implementation}
\label{sec:implementation}

\subsection{Coherence and effectiveness of the work plan}
\label{sec:implementation_work_plan}

The proposal should be designed in such a way to achieve the desired impact. 
A Gantt Chart should be included in the text listing the following:

\begin{itemize}
  \item Work Packages titles (for EF there should be at least 1 WP); 
  \item List of major deliverables, if applicable;%
  \footnote{A deliverable is a distinct output of the action, meaningful in terms of the action's overall objectives and may be a report, a document, a technical diagram, a software, etc.
  Should the applicants wish to participate in the pilot on Open Research Data, the Data Management Plan should be indicated here.\\
  Deliverable numbers ordered according to delivery dates. 
  Please use the numbering convention <WP number>.<number of deliverable with that WP>. 
  For example, deliverable 4.2 would be the second deliverable from work package 4.}
  \item List of major milestones, if applicable;%
  \footnote{Milestones are control points in the action that help to chart progress. 
  Milestones may correspond to the completion of a key deliverable, allowing the next phase of the work to begin.
  They may also be needed at intermediary points so that, if problems have arisen, corrective measures can be taken. 
  A milestone may be a critical decision point in the action where, for example, the researcher must decide which of several technologies to adopt for further development.}
  \item Secondments, if applicable.
\end{itemize}

\noindent
The schedule should be in terms of number of months elapsed from the start of the action.




\begin{figure}[!htbp]
\begin{center}

\begin{ganttchart}[
    canvas/.append style={fill=none, draw=black!5, line width=.75pt},
    hgrid style/.style={draw=black!5, line width=.75pt},
    vgrid={*1{draw=black!5, line width=.75pt}},
    title/.style={draw=none, fill=none},
    title label font=\bfseries\footnotesize,
    title label node/.append style={below=7pt},
    include title in canvas=false,
    bar label font=\small\color{black!70},
    bar label node/.append style={left=2cm},
    bar/.append style={draw=none, fill=black!63},
    bar progress label font=\footnotesize\color{black!70},
    group left shift=0,
    group right shift=0,
    group height=.5,
    group peaks tip position=0,
    group label node/.append style={left=.6cm},
    group progress label font=\bfseries\small
  ]{1}{24}
  \gantttitle[
    title label node/.append style={below left=7pt and -3pt}
  ]{Month:\quad1}{1}
  \gantttitlelist{2,...,24}{1} \\
  \ganttgroup{Work Package}{1}{24} \\
  \ganttgroup{Deliverable}{24}{24} \\
  \ganttgroup{Milestone}{5}{5} \\
  \ganttgroup{Secondment}{20}{23} \\
  \ganttgroup{Short stay}{16}{16} \\
  \ganttgroup{Training}{5}{5} \\
  \ganttgroup{Dissemination}{23}{24} \\
  \ganttgroup{Communication}{12}{12} \\
  \ganttgroup{Other}{18}{21}
\end{ganttchart}

\end{center}
\caption{Example Gantt Chart}
\end{figure}




\subsection{Appropriateness of the allocation of tasks and resources}
\label{sec:implementation_resources}

Describe how the work planning and the resources mobilised will ensure that he research and training objectives will be reached.

\medskip\noindent
Explain why the amount of person-months is appropriate in relation to the activities proposed.





\subsection{Appropriateness of the management structure and procedures, including risk management}
\label{sec:implementation_management}

Describe the: 

\begin{itemize}
  \item Organization and management structure, as well as the progress monitoring mechanisms put in place, to ensure that objectives are reached
  \item Research and/or administrative risks that might endanger reaching the action objectives and the contingency plans to be put in place should risk occur  
  \item  Involvement of entity with a capital or legal link to the beneficiary (in particular, name of the entity, type of link with the beneficiary and tasks to be carried out), if applicable
\end{itemize}





\subsection{Appropriateness of the institutional environment (infrastructure)}
\label{sec:implementation_infrastructure}

The active contribution of the beneficiary to the research and training activities should be described. 
For Global Fellowships the role of partner organisations in Third Countries for the outgoing phase should appear. 

\begin{itemize}
  \item Give a description of the main tasks and commitments of the beneficiary and all partner organisations (if applicable).
  \item Describe the infrastructure, logistics, facilities offered in as far they are necessary for the good implementation of the action.
\end{itemize}





\markEndPageLimit

%%% Local Variables:
%%% mode: latex
%%% TeX-master: "IF-2018-Part_B"
%%% End:

% !TEX root = ./IF-2018-Part_B.tex

\newpage
\addcontentsline{toc}{section}{\hspace{-0.5cm}Document 2}
\section{CV of the Experienced Researcher}
\label{sec:cv}

The CV is intrinsic to the evaluation of the whole proposal and is assessed throughout the 3 evaluation criteria by the expert evaluators.
Please make sure that the information between part A and B is fully consistent.

\medskip\noindent
Applicants without a doctorate should clearly justify any period of Full-Time
Equivalent Research Experience in the CV part B (section~\ref{sec:cv}). It is essential that the
CV clearly explains how the Research Experience is calculated, following this
template.

\medskip\noindent
This section should be limited to maximum 5 pages and should include \textbf{the standard academic and research record}. 
Any research career gaps and/or unconventional paths should be clearly explained so that this can be fairly assessed by the independent evaluators.

\medskip\noindent
The {\em experienced researchers} must provide a list of achievements reflecting their track record, if applicable:

\begin{enumerate}
\item \textbf{Publications} in peer-reviewed scientific journals, peer-reviewed conference proceedings and/or monographs of their respective research fields, indicating also the number of citations (excluding self-citations) they have attracted.
\item Granted \textbf{patent(s)}.
\item \textbf{Research monographs, chapters} in collective volumes and any translations thereof.
\item \textbf{Invited presentations} to peer-reviewed, internationally established conferences and/or international advanced schools.
\item \textbf{Research expeditions} led by that the {\em experienced researcher}. 
\item \textbf{Organisation of International conferences} in the field of the researcher (membership in the steering and/or programme committee).
\item Examples of \textbf{participation in industrial innovation}.
\item \textbf{Prizes and Awards}.
\item \textbf{Funding} received so far.
\item \textbf{Supervising} and \textbf{mentoring} activities.
\end{enumerate}





\newpage
\section{Capacity of the Participating Organisations}
\label{sec:capacities}

Beneficiaries and partner organisations must complete the appropriate table below.

\medskip\noindent
Complete one table (min font size: 9) of maximum \ul{one page per beneficiary and one page per partner organisation}. 
The expert evaluators will be instructed to disregard content above this limit.
\vspace{\baselineskip}


\begin{table}[h!]
{\fontsize{9bp}{1em}\selectfont % should be 9pt
\noindent\begin{tabular}{|>{\raggedright}p{.25\textwidth}|p{.7\textwidth}|}\hline
  \multicolumn{2}{|l|}{\cellcolor{gray!50}\textbf{Beneficiary X}} \\\hline
\textbf{General Description} &

\\\hline
\textbf{Role and Profile of key persons (supervisor)} &
{\em (names, title, qualifications of the main supervisor)}
{\em }
\\\hline
\textbf{Key Research Facilities, Infrastructure and Equipment} &
{\em Demonstrate that the beneficiary has sufficient facilities and infrastructure to host and/or offer a suitable environment for training and transfer of knowledge to the recruited experienced researcher.
\newline
If applicable, indicate the name of the entity with a capital or
legal link to the beneficiary and its role in the action.
}
\\\hline
\textbf{Independent research premises?} &
{\em Please explain the status of the beneficiary's research facilities\----i.e. are they owned by the beneficiary or rented by it? Are its research premises wholly independent from other entities?
\newline
If applicable, indicate the name of the entity with a capital or
legal link to the beneficiary and describe the nature of the
link.
}
\\\hline
\textbf{Previous Involvement in Research and Training Programmes} &
{\em Detail any (maximum 5) relevant EU, national or international research and training actions/projects in which the beneficiary has previously participated.}
\\\hline
\textbf{Current involvement in Research and Training Programmes} &
{\em Detail the EU and/or national research and training actions in which the beneficiary is currently participating.}
\\\hline
\textbf{Relevant Publications and/or research/innovation products} &
{\em (Max 5) Only list items (co-)produced by the supervisor}
\\\hline
  \multicolumn{2}{|l|}{\cellcolor{gray!50}\textbf{Partner Organisation Y}} \\\hline
\textbf{General description} &
\\\hline
\textbf{Key Persons and Expertise (supervisor)} &
\\\hline
\textbf{Key Research facilities, infrastructure and equipment} &
\\\hline
\textbf{Previous and Current Involvement in Research and Training Programmes} &
\\\hline
\textbf{Relevant Publications and/or research/innovation product} &
{\em (Max 3)}
\\\hline
\end{tabular}}
\end{table}




\newpage
\section{Ethical Issues}
\label{sec:ethics}

Compliance with the relevant ethics provisions is essential from the beginning to the end of the action and is an integral part of research funded by the European Union within Horizon 2020. 

\medskip\noindent
Applicants submitting research proposals for funding with Marie Sk\l{}odowska-Curie actions in Horizon 2020 should demonstrate proactively that they are aware of and will comply with European and national legislation and fundamental ethical principles, including those reflected in the \href{http://www.europarl.europa.eu/charter/pdf/text_en.pdf}{Charter of Fundamental Rights of the European Union} and the \href{http://www.echr.coe.int/Documents/Convention_ENG.pdf}{European Convention on Human Rights and its Supplementary Protocols}.

\medskip\noindent
Please be aware that it is the applicants' responsibility to identify any potential ethical issue, 
to handle the ethical aspects of the proposal and to detail how these aspects will be addressed.

\bigskip\noindent
{\large {\bf \ul{The Ethics Review Procedure in Horizon 2020}}}

\medskip\noindent
All proposals above threshold and considered for funding will undergo an Ethics Review carried out by independent ethics experts. 
When submitted a proposal to Horizon 2020, all applicants are required to complete an ``{\bf Ethics Issues Table (EIT)}'' in the Part A of the proposal. 
Applicants who flag ethical issues in the EIT have to also complete a more in-depth {\bf Ethics Self-Assessment in Part B.}

\medskip\noindent
The ethics self-assessment will become part of the Grant Agreement and may thus lead to binding obligations that may later on be checked by ethics checks, reviews and audits.

\medskip\noindent
For more details, please refer to the H2020 \href{http://ec.europa.eu/research/participants/data/ref/h2020/grants_manual/hi/ethics/h2020_hi_ethics-self-assess_en.pdf}{``How to complete your Ethics Self-Assessment''} guide.

\bigskip\noindent
{\large {\bf \ul{Ethics Self-Assessment (Part B)}}}

\medskip\noindent
The Ethics Self-Assessment must:

{\bf
\begin{enumerate}[leftmargin=*, label=\arabic*)]
  \item Describe how the proposal meets the EU and national legal and ethics requirements of the country/countries where the task raising ethical is to be carried out.
\end{enumerate}
}

\medskip\noindent
For more information on how to deal with Third Countries (in the context of ethics
appraisal, Third Country refers to non-EU country; Associated Countries are "ethics"
TC) please see Article 34 of the \href{http://ec.europa.eu/research/participants/data/ref/h2020/grants_manual/amga/h2020-amga_en.pdf}{Annotated Model Grant Agreement},
as well as the following \href{http://ec.europa.eu/justice/data-protection/international-transfers/adequacy/index_en.htm}{link}.
Please ensure and confirm that the research performed outside the EU
is compatible with the Union, National and International legislation and could have
been legally conducted in one of the EU Member States.

\medskip\noindent
Please list the documents provided with their expiry date.

\medskip\noindent
Ensure early compliance of the proposed research with EU and national legislation on ethics in research.
Should your proposal be selected for funding, 
you will be required to confirm that you have obtained the following documents (if applicable):

\begin{enumerate}[label=(\alph*)]
  \item any ethics committee opinion required under national law and
  \item any notification or authorisation for activities raising ethical issues required under national and/or European law
\end{enumerate}

\noindent
needed for implementing the action tasks in question.

\medskip\noindent
If you have not already applied for/received the ethics approval/required ethics documents when submitting the proposal, 
please indicate in this section the approximate date when you will obtain the missing approval/any other ethics documents. 
Please state explicitly that you will not proceed with any research with ethical implications before 
obtaining the necessary authorizations/opinions.

\medskip\noindent
\fcolorbox{black}{gray!50}{\parbox{\textwidth}{
{\em The documents must be kept on file and be submitted upon request by the beneficiary
to the Agency (see Article 52). If they are not in English, they must be submitted
together with an English summary, which shows that the action tasks in question are
covered and includes the conclusions of the committee or authority concerned (if
available).}

\medskip\noindent
{\em If you plan to request these ethics documents specifically for your} proposed action\em{, your request must contain an explicit reference to the} action\em{'s title.}
}}

\bigskip
{\bf 
\begin{enumerate}[leftmargin=*, label=\arabic*), start=2]
  \item Explain in detail how you intend to address the ethical issues flagged, in particular with regard to: 
  \begin{itemize}
    \item {\normalfont the research {\bf objectives} (e.g., study of vulnerable populations, cooperation with a Third Country, etc.);}
    \item {\normalfont the research {\bf methodology} (e.g., clinical trials, involvement of children and related information and consent/assent procedures, data protection and privacy issues related to data collected, etc.);}
    \item {\normalfont the potential {\bf impact} of the research (e.g. dual use issues, environmental damage, malevolent use, etc.);}
    \item {\normalfont appropriate health and safety procedures - conforming to relevant local/national guidelines/legislation - for the staff involved;}
    \item {\normalfont possible harm to the environment the research might cause, (as an example: environmental risks of nanomaterials), and measures that will be taken to mitigate the risks.}
  \end{itemize}
\end{enumerate}
}




\newpage
\section{Letter of Commitment (GF only)}
\label{sec:letters}

For the Global Fellowship proposals, a letter of Commitment \textbf{of the partner
organisations} (hosting the outgoing phase in a third country) must be included in part
B-2 to ensure their real and active participation. these should not be attached as a
separate PDF file or as an embedded file since this makes them invisible.

\medskip\noindent
GF Proposals which fail to include a letter of commitment of the partner organisation
will be declared \textbf{inadmissible}.

\medskip\noindent
Minimum requirements for the letter of commitment: 

\begin{itemize}
  \item \textbf{heading} or \textbf{stamp} from the institution; 
  \item up-to-date (may not be dated prior to the call publication); 
  \item the text must demonstrate the will to actively participate in the (identified) proposed action and the precise role;
\end{itemize}

\noindent
Please note that no template for these letters is provided, only general rules.





\newpage
\label{sec:endpage}
\vspace{15mm}
\begin{center}


        \Large{
      
     
        \textbf{ENDPAGE}
  
          \vspace{15mm}
          MARIE SK\L{}ODOWSKA-CURIE ACTIONS\\
          \vspace{1cm}
          
          \textbf{\acf{IF}}\\
          \textbf{Call: H2020-MSCA-IF-2018}
          \vspace{2cm}                   

          PART B
          \vspace{2.5cm}

          ``{\sc \ac{PropAcronym}\xspace}''
          \vspace{2cm}

          \textbf{This proposal is to be evaluated as:}
          \vspace{.5cm}

          \textbf{[EF-ST] [EF-CAR] [EF-RI] [EF-SE] [GF]}\\
        }
        \large{[Delete as appropriate]}

  \end{center}
\vspace{1cm}

%%% Local Variables:
%%% mode: latex
%%% TeX-master: "IF-2018-Part_B"
%%% End:

%\end{linenumbers}
\end{document}


